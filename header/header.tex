% 
%Schriftgrösse, Layout, Papierformat, Art des Dokumentes
\documentclass[10pt,twoside,a4paper,fleqn]{article}
%Einstellungen der Seitenränder
\usepackage[left=0.5cm,right=0.5cm,top=0.5cm,bottom=0.5cm,includeheadfoot]{geometry}

% Sprache, Zeichensatz, packages ======================================================================
\usepackage[utf8]{inputenc} % Zeichenkodierung: UTF-8 (für Umlaute)
\usepackage[ngerman]{babel,varioref} % Deutsche Sprache
\usepackage{amssymb}
\usepackage{amsmath}
\usepackage{fancybox}
\usepackage{graphicx}
\usepackage{color}
\usepackage{lastpage}
\usepackage{wrapfig}
\usepackage{fancyhdr}
\usepackage{hyperref}
\usepackage{verbatim}
\usepackage{pdfpages}
\usepackage[T1]{fontenc}
\usepackage{multicol} %Spalten, für integrale.tex
\usepackage{rotating} % Rotating table, für tabellen.tex; [Richtung der Rotation] => [figuresright] oder [figuresleft]
%\usepackage{wasysym}			% Blitz
%\usepackage{esint}				% erweiterte Integralsymbole
\usepackage{extarrows}			% Packet für langer Pfeil mit Text


\usepackage{tabularx}
%Farben für tabelle
\usepackage{color, colortbl}
\definecolor{Gray}{gray}{0.9}
\definecolor{LightCyan}{rgb}{0.88,1,1}


%Eigene Befehle =======================================================
%Custom functions 
\DeclareMathOperator{\arccot}{arccot}  %für arccot

%definiere Befehl:
\newcommand{\formelbuch}[1]{\textcolor{red}{\textit{#1}}}	%Verweis auf Seite im Formelbuch


%%Subsections
%
%\usepackage{titlesec}
%
%\setcounter{secnumdepth}{4}
%
%\titleformat{\paragraph}
%{\normalfont\normalsize\bfseries}{\theparagraph}{1em}{}
%\titlespacing*{\paragraph}
%{0pt}{3.25ex plus 1ex minus .2ex}{1.5ex plus .2ex}

%Layout
\raggedbottom


%pdf info
\hypersetup{pdfauthor={\authorinfo},pdftitle={\titleinfo},colorlinks=false}
%linkbordercolor=white
\author{\authorinfo}
\title{\titleinfo}

%Kopf- und Fusszeile
\pagestyle{fancy}
\fancyhf{}
%Linien oben und unten
\renewcommand{\headrulewidth}{0.5pt} 
\renewcommand{\footrulewidth}{0.5pt}


%Kopfzeilen
\fancyhead[R]{Seite \thepage { }von \pageref{LastPage}}
\fancyhead[L]{\titleinfo{ }\tiny{\textit{(\versioninfo)}}}
%Fusszeilen
\fancyfoot[L]{\footnotesize{\authorinfo}}
\fancyfoot[C]{\footnotesize{\licence \quad $\rightarrow$ \href{https://github.com/HSR-Stud}{Github: HSR-Stud}}}
\fancyfoot[R]{\footnotesize{\today}}

\setlength{\parindent}{0pt}



% Custom Commands ===============================================
%\renewcommand{\thesubsection}{\arabic{subsection}}
\newcommand{\me}[1]{\ensuremath{\left\{#1\right\}}}
\newcommand{\dme}[2]{\ensuremath{\left\{#1\,\vert\,#2 \right\}}}
\newcommand{\abs}[1]{\ensuremath{\left\vert#1\right\vert}}
\newcommand{\un}[1]{\; \unit{#1} }
\newcommand{\unf}[2]{\;\left[ \unitfrac{#1}{#2} \right]}
\newcommand{\norm}[2][\relax]{\ifx#1\relax \ensuremath{\left\Vert#2\right\Vert}\else \ensuremath{\left\Vert#2\right\Vert_{#1}}\fi}
\newcommand{\enbrace}[1]{\ensuremath{\left(#1\right)}}
\newcommand{\nira}[1]{\ensuremath{\overset{n \rightarrow \infty}{\longrightarrow}}}
\newcommand{\os}[2]{\ensuremath{\overset{#1}{#2}}}
\makeatletter
\newcommand{\Ra}[0]{\ensuremath{\Rightarrow}}
\newcommand{\ra}[0]{\ensuremath{\rightarrow}}
\newcommand{\gk}[1]{\ensuremath{\left\lfloor#1\right\rfloor}}
\newcommand{\sprod}[2]{\ensuremath{%
		\setbox0=\hbox{\ensuremath{#2}}
		\dimen@\ht0
		\advance\dimen@ by \dp0
		\left\langle #1\rule[-\dp0]{0pt}{\dimen@},#2\right\rangle}}

\newcommand{\tab}[1][1cm]{\hspace*{#1}}

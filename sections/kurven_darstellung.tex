\section{Kurven}

%TODO:4 Arten wie man das Integral berechnet. Siehe: Analysis für Dummies


\subsection{Darstellungsarten von Kurven \formelbuch{49ff}} %Seitenzahlen OK 10 Auflage
  \begin{minipage}[t]{3.5cm}
    Funktion (explizit) \\
    $ y = f(x)$ \\
        \tiny{(Bronstein Form 2.4)}
  \end{minipage}
  \begin{minipage}[t]{6cm}    
    Koordinatengleichung (implizit) \\
    $ F(x,y) = 0 $ \\
        {\tiny{(Bronstein Form 2.5)}}\\
        \\
    $ F(\varphi,r) = 0 $ 
  \end{minipage}
  \begin{minipage}[t]{5.5cm}    
    Parameterform \\
    $ \left( \begin{array} {l} x(t) \\ y(t) \end{array} \right) =
          \left( \begin{array} {l} \Psi(t) \\ \varphi(t) \end{array} \right)$\\
        {\tiny{(Bronstein Form 2.6)}}\\
     \\
     $ \left(
     \begin{array}{l} 
	    r = g(t) \\
	    \varphi = h(t)
     \end{array} 
     \right)$\\
     \\
     {\small{mit $t \in \mathbb{I}$, wobei $\mathbb{I}$ ein Intervall ist}}
  \end{minipage} 
  \begin{minipage}[t]{3cm}
      Polarform x\\
      $ r=f(\varphi) mit \varphi \in \mathbb{D}_{\textit{f}} $ \\
        {\tiny{(Bronstein Form 3.427)}}
    \end{minipage}\\

 % \textit{Hat man die explizite Form gegeben, so hat man automatisch die Implizite- und Parameter-Form}

\subsection{Umrechnen diverser Systeme \formelbuch{197}} %Seitenzahlen OK 10 Auflage
\renewcommand{\arraystretch}{1}
\begin{tabular}{|ll|l|}
	\hline 
von & nach & Umrechnung \\  \hline \hline
Parameter 
  & $\Rightarrow$ explizit
  & $t = f(x) \qquad y = g(f(x))$\\	\hline
Ex- bzw. implizit 
  & $\Rightarrow$ Polar
  &  Ersetze: \quad $ x=r \cdot \cos(\varphi) \qquad y=r \cdot \sin(\varphi) \qquad x^2 + y^2=r^2$\\ \hline
Polar 
  & $\Rightarrow$ implizit
  & Ersetze: \quad $r \cdot \cos(\varphi)=x \qquad r \cdot \sin(\varphi)=y \qquad r=\sqrt{x^{2}+y^{2}}$\\ \hline
Polar
  & $\Rightarrow$ Parameterform
  & $\left( \begin{array} {l} x(\varphi) \\ y(\varphi) \end{array} \right) =
          \left( \begin{array} {l} r(\varphi) \cos(\varphi) \\ r(\varphi) \sin(\varphi) \end{array}
          \right) = \left( \begin{array} {l} r \cdot \cos(\varphi) \\ r \cdot \sin(\varphi) \end{array}
          \right)$ \\	\hline
Explizit
  & $\Rightarrow$ Parameter
  & $\left( \begin{array} {l} x(t) \\ y(t) \end{array} \right) =
          \left( \begin{array} {l} x(t) \\ t \end{array}
          \right)$ \\ \hline
Einzelner Punkt  
  & $\Rightarrow$ Polar
  & $ r = \sqrt{x^2 + y^2} \qquad
  \varphi = \left.
  		 \begin{cases} 
  			\arctan \left(\frac{y}{x}\right) + \pi   &x < 0\\
             \arctan \left(\frac{y}{x}\right)   & x > 0\\
             \frac{\pi}{2}      & x = 0;\; y > 0\\
             -\frac{\pi}{2}     & x = 0;\; y < 0\\
             \text{unbestimmt}    & x = y = 0
          \end{cases} 
            \right\} \quad  = \quad
          \begin{cases}
             \arccos \left(\frac{x}{r}\right) & x \geqq 0 \\
             -\arccos \left(\frac{x}{r}\right) & x<0 
          \end{cases} $ \\ \hline
\end{tabular}